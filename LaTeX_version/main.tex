% Copyright 2004 by Till Tantau <tantau@users.sourceforge.net>.
%
% In principle, this file can be redistributed and/or modified under
% the terms of the GNU Public License, version 2.
%
% However, this file is supposed to be a template to be modified
% for your own needs. For this reason, if you use this file as a
% template and not specifically distribute it as part of a another
% package/program, I grant the extra permission to freely copy and
% modify this file as you see fit and even to delete this copyright
% notice.

\documentclass[aspectratio=169]{beamer}

\usepackage{textcomp}
\usepackage{array} % needed for \arraybackslash
\usepackage{graphicx}
\usepackage{adjustbox} % for \adjincludegraphics
\usepackage{wrapfig}




% There are many different themes available for Beamer. A comprehensive
% list with examples is given here:
% http://deic.uab.es/~iblanes/beamer_gallery/index_by_theme.html
% You can uncomment the themes below if you would like to use a different
% one:
%\usetheme{AnnArbor}
%\usetheme{Antibes}
%\usetheme{Bergen}
%\usetheme{Berkeley}
%\usetheme{Berlin}
%\usetheme{Boadilla}
%\usetheme{boxes}
%\usetheme{CambridgeUS}
%\usetheme{Copenhagen}
%\usetheme{Darmstadt}
%\usetheme{default}
\usetheme{Frankfurt}
%\usetheme{Goettingen}
%\usetheme{Hannover}
%\usetheme{Ilmenau}
%\usetheme{JuanLesPins}
%\usetheme{Luebeck}
%\usetheme{Madrid}
%\usetheme{Malmoe}
%\usetheme{Marburg}
%\usetheme{Montpellier}
%\usetheme{PaloAlto}
%\usetheme{Pittsburgh}
%\usetheme{Rochester}
%\usetheme{Singapore}
%\usetheme{Szeged}
%\usetheme{Warsaw}

\title{Systematic Study of Background Subtraction Techniques for EELS}

% A subtitle is optional and this may be deleted
%\subtitle{Optional Subtitle}

%\author{V.C.~Angadi\inst{1} \and T.~Walther\inst{1}}
\author{V.C.~Angadi \and T.~Walther}
% - Give the names in the same order as the appear in the paper.
% - Use the \inst{?} command only if the authors have different
%   affiliation.

\institute[University of Sheffield] % (optional, but mostly needed)
{
%  \inst{1}%
  Department of Electronic and Electrical Engineering\\
  University of Sheffield,\\
  Sheffield, UK
%  \and
%  \inst{2}%
%  Department of Theoretical Philosophy\\
%  University of Elsewhere
}
% - Use the \inst command only if there are several affiliations.
% - Keep it simple, no one is interested in your street address.

\date{EMAG 2016 - Durham}
% - Either use conference name or its abbreviation.
% - Not really informative to the audience, more for people (including
%   yourself) who are reading the slides online

\subject{Electron Energy-loss Spectroscopy}
% This is only inserted into the PDF information catalog. Can be left
% out. 

% If you have a file called "university-logo-filename.xxx", where xxx
% is a graphic format that can be processed by latex or pdflatex,
% resp., then you can add a logo as follows:

%\pgfdeclareimage[height=0.5cm]{logo}{UniversityofSheffield}
%\logo{\pgfuseimage{logo}}
% or
% logo of my university
%\titlegraphic{\includegraphics[width=4cm]{logo}}

% Delete this, if you do not want the table of contents to pop up at
% the beginning of each subsection:
%\AtBeginSubsection[]
%{
%  \begin{frame}<beamer>{Outline}
%    \tableofcontents[currentsection,currentsubsection]
%  \end{frame}
%}

% Let's get started
\begin{document}

\begin{frame}
  \titlepage
\end{frame}

%\begin{frame}{Outline}
%  \tableofcontents
  % You might wish to add the option [pausesections]
%\end{frame}

% Section and subsections will appear in the presentation overview
% and table of contents.
\section{Motivation}

%\subsection{First Subsection}

\begin{frame}{Motivation}%{Optional Subtitle}
  \begin{itemize}
  \item EELS is a very important tool for elemental analysis of materials science.
  \item Net core loss = spectrum \textminus ~background.
  \item extrapolating the background in the presence of preceding edges is difficult.
  \item Background fitting in the presence of ELNES, EXELFS etc.
  \item Explore extrapolation from other regions such as post-ionization edge.
  \item Explore extrapolations which are combination of fits in different region.
  \end{itemize}
\end{frame}

%\subsection{Second Subsection}

% You can reveal the parts of a slide one at a time
% with the \pause command:
%\begin{frame}{Second Slide Title}
%  \begin{itemize}
%  \item {
%    First item.
%    \pause % The slide will pause after showing the first item
%  }
%  \item {   
%    Second item.
%  }
%  % You can also specify when the content should appear
%  % by using <n->:
%  \item<3-> {
%    Third item.
%  }
%  \item<4-> {
%    Fourth item.
%  }
%  % or you can use the \uncover command to reveal general
%  % content (not just \items):
%  \item<5-> {
%    Fifth item. \uncover<6->{Extra text in the fifth item.}
%  }
%  \end{itemize}
%\end{frame}

\section{Pre-edge region}

%\subsection{Another Subsection}

\begin{frame}{Pre-edge region}

\begin{columns}[t]
	\begin{column}{.6\textwidth}
		\begin{itemize}
			\item Background crossing the spectrum -unphysical.
			\item As-L$_{2,3}$ edge can still be quantified by integrating only the positive core-loss region.
			\item This yields an under-estimate.
			\item The Ga-L$_{2,3}$ edge is straight forward as the it has very large pre-edge region.
			\item Highly associated with large systematic errors for large integration range.
			\item Systematic errors are difficult to identify by regression and quantification.
		\end{itemize}
	\end{column}

	\begin{column}{.4\textwidth}
		\adjincludegraphics[width=\linewidth,valign=t]{As_pre_p}
	\end{column}
\end{columns}

\end{frame}

%\begin{frame}{Blocks}
%\begin{block}{Block Title}
%You can also highlight sections of your presentation in a block, with it's own title
%\end{block}
%\begin{theorem}
%There are separate environments for theorems, examples, definitions and proofs.
%\end{theorem}
%\begin{example}
%Here is an example of an example block.
%\end{example}
%\end{frame}


\section{Post-edge region}

\begin{frame}{Post-edge region}

\begin{columns}[t]
	\begin{column}{.6\textwidth}
		\begin{itemize}
			\item Extrapolate from end of the spectrum and offset vertically to cross through the edge onset.
			\item Over-estimate of the core-loss edge intensity.
			\item Poissonian statistical error bars are very large.
			\item Highly associated with statistical errors and are difficult to identify by regression and quantification.
			\item For Ga-L$_{2,3}$ edge, As-L$_{2,3}$ edge is subtracted to provide larger post-edge region for extrapolation.
		\end{itemize}
	\end{column}

	\begin{column}{.4\textwidth}
		\adjincludegraphics[width=\linewidth,valign=t]{As_post_p}
	\end{column}
\end{columns}

\end{frame}

%\begin{frame}{Blocks}
%\begin{block}{Block Title}
%You can also highlight sections of your presentation in a block, with it's own title
%\end{block}
%\begin{theorem}
%There are separate environments for theorems, examples, definitions and proofs.
%\end{theorem}
%\begin{example}
%Here is an example of an example block.
%\end{example}
%\end{frame}

\section{Optimal fit}

\begin{frame}{Optimal fit}

\begin{columns}[t]
	\begin{column}{.6\textwidth}
		\begin{itemize}
			\item Fits in pre-edge and post-edge regions provide under-and over-estimate of the core-loss edges.
			\item Only backgrounds which are physically meaningful are retained.
			\item Yields positive core-loss.
			\item Have smaller error bars.
			\item More discussions in poster \textbf{\LARGE{\alert{P16}}}.
		\end{itemize}
	\end{column}

	\begin{column}{.4\textwidth}
		\adjincludegraphics[width=0.78\linewidth,valign=t]{As_opti_p}
		\vspace{0.1cm}
		\adjincludegraphics[width=0.78\linewidth,valign=t]{histogram}
	\end{column}
\end{columns}

\end{frame}


% Placing a * after \section means it will not show in the
% outline or table of contents.
%\section*{Summary}

%\begin{frame}{Summary}
%  \begin{itemize}
%  \item
%    The \alert{first main message} of your talk in one or two lines.
%  \item
%    The \alert{second main message} of your talk in one or two lines.
%  \item
%    Perhaps a \alert{third message}, but not more than that.
%  \end{itemize}
%  
%  \begin{itemize}
%  \item
%    Outlook
%    \begin{itemize}
%    \item
%      Something you haven't solved.
%    \item
%      Something else you haven't solved.
%    \end{itemize}
%  \end{itemize}
%\end{frame}



% All of the following is optional and typically not needed. 
%\appendix
%\section<presentation>*{\appendixname}
%\subsection<presentation>*{For Further Reading}

%\begin{frame}[allowframebreaks]
%  \frametitle<presentation>{For Further Reading}
    
%  \begin{thebibliography}{10}
    
%  \beamertemplatebookbibitems
  % Start with overview books.

%  \bibitem{Author1990}
%    A.~Author.
%    \newblock {\em Handbook of Everything}.
%    \newblock Some Press, 1990.
 
    
%  \beamertemplatearticlebibitems
  % Followed by interesting articles. Keep the list short. 

%  \bibitem{Someone2000}
%    S.~Someone.
%    \newblock On this and that.
%    \newblock {\em Journal of This and That}, 2(1):50--100,
%    2000.
%  \end{thebibliography}
%\end{frame}

\end{document}


